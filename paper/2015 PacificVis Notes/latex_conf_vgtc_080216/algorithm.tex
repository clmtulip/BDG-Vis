\section{Laplacian based Dynamic Graph Drawing Algorithm}
In this section we describe our algorithm in detail. 
Given a sequence of online graphs $G_{0}, ..., G_{n}$, where $G_{0} = (V_{0}, E_{0}), ..., G_{n} = (V_{n}, E_{n})$,
the goal of our algorithm is to produce a sequence of layouts $L_{0}, ..., L_{n}$, where $L_{i}$ is a straight line
drawing of $G_{i}$. We first describe how to compute the online dynamic layout $L_{i}, i \ge 1$, when given $L_{i-1}$
and $G_{i}$. Then, we discuss the way to compute initial layout $L_{0}$.

\subsection{Laplacian based dynamic graph layout algorithm}
In the dynamic graph sequence, for each graph $G_{i} = (V_{i}, E_{i})$, each node has a unique id 'UId'.
Through this unique id, we can recognize the same nodes in different time step.
The following are the basic steps of our algorithm:
\begin{enumerate}
    \item Calculate initial node positions from previous graph layout $L_{i-1}$.
    \item Run force-directed layout algorithm.
    \item Run Laplacian Constrained Distance Embedding algorithm to refine the mental map.
\end{enumerate}

\textbf{Step 1:} For each node $v$ in $G_{i}$, which is existed in $G_{i-1}$, copy
the coordinates from $L_{i-1}$ to this node; for those new added nodes, calculate their coordinates
according to their neighbors as follows: if node $v$ has more than one positioned neighbors, then 
this node is assigned the barycenter of all its positioned neighbours; if node $v$ has only one positioned
neighbour $p$, select a random position on the circle which take $p$ as center of this circle, and the average edge length 'avg_edge_length' of $L_{i-1}$ as radius; if node $v$ has no positioned neighbour, then it can be placed
randomly.

\textbf{Step 2:} Run force-directed layout algorithm starting from the positions assigned in step 1. Here we use 
the simulated annealing force-directed layout algorithm~\cite{Davidson:1996:ATG} to adjust the layout $L_{i}$. For
different nodes we set different cooling parameters. In order to keep the existed nodes as stable as possible, we
have to frozen these nodes which copy coordinates from $L_{i-1}$, only small movements allowed. For those nodes that
are totally random positioned, we should let them move far away enough to find their best positions, so the cooling 
parameter should be high. In our implementation, scores of 0.25, 0.5, 0.75 and 1 are assigned to nodes positioned
copied from $L_{i-1}$, at the barycenter of two or more neighbors, according to one neighbor, and totally random ones,
respectively. 

\textbf{Step 3:} Although we have preserved user mental map by setting a small parameter of cooling step, but it's still far from stable. The nodes both existed in graph $G_{i-1}$ and $G_{i}$ move a lot by the forces from 
new added and disappeared nodes and edges. Inspired by the Laplacian constrained graph layout algorithm~\cite{Yuan:2012:TVCG}, we can take the subgraph consist of nodes that are copied from $L_{i-1}$ as
a user input subgraph, then run the Laplacian constrained distance embedding algorithm. 

\subsection{Initial layout of dynamic graph}

