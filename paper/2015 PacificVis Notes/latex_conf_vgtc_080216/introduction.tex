\section{Introduction}
In many applications,
graphs are widely used to represent social connections, physical networks, or other relationships.
Most of these graphs are changing over time, for example in a social connection, some new people add
in, some old ones leave. To reflect the evolution of the organization or system represented by the graph, 
we can keep a track of snapshots once the graph changed. This kind of sequence graph layout is called
dynamic graph layout problem.

Traditional Graph drawing algorithms are well studied since
early year, layout methods, such as force-directed or stress model layout algorithms are
all concentrate on static graph. They produce a totally new layout each time we run the algorithm. 
Since the dynamic graph sequence is a evolution of a graph, there must be some parts not changed, which
we hope their layout are also not changed. Therefore in dynamic graph drawing, the challenge is that we have to 
layout a current graph aesthetically good while preserving the "mental map" of previous time step.

Mental map was first proposed as a concept in~\cite{Eades:1991:POC}, it means a abstract structural information
a user forms by looking at the layout of a graph. The mental map facilitates memory of a graph or comparison between
it and other graph layouts. Preserving the mental map helps users to quickly recognize the new graph, which
is important in dynamic graph drawing.  


